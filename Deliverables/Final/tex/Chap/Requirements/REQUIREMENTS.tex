% CHAPTER - Requirements Elicitation and Specification definition ------------
\chapter{Requirements Elicitation and Specifications Definition}%
\label{ch:requirements-specs}
In this stage the project requirements are elicited, identifying the key
requirements and constraints the system being developed must meet from the
end-user perspective, captured in natural language in a product requirements
document. The end-user perspective is generally abstract, thus requiring a
methodic approach to obtain well-defined product requirements, i.e., product
specifications. The product specifications are the result of a compromise
between end-user requirements and its feasibility within the available project
resources (time, budget, and technologies available). As the specifications are
well-defined, they serve as design guidelines for the development team and can
be tested later on to assess its feasibility and, ultimately, the quality of the product.
%
%
\section{Foreseen specifications}%
\label{sec:fores-spec}
\section{Foreseen product specifications}
\label{sec:org31f7574}
The foreseen product specifications are listed as topics below (sketch).

\subsection{Autonomy}
\label{sec:org7364ba5}
The vehicle is operated off-the-grid, thus, a portable power source must be included. The autonomy referes to the time interval between battery fully charged and safely discharged and should be observed for the following scenarios:
\begin{itemize}
\item No load;
\item Vehicle operating at maximum speed;
\item Vehicle operating at minimum speed.
\end{itemize}
\subsection{Velocity}
\label{sec:org08718bc}
The velocity the car achieves is determined by the voltage of the motors, allowing to operate the car at the desired velocity simply varying the dutycicle of the control signal to the motors. The maximum velocity is achieved when the dutycicle of all motor control signals are 1 and the car does not have any load. 
\subsection{Safety}
\label{sec:org83942c3}
For a remote controlled car, safety concerns not only the car itself and all of the equipment, but also the humans that interact with the car:
\begin{itemize}
\item Car: If the user issues a command that would cause damage to the system, the
system should take corrective measures to prevent it. The same holds true if
the communication between user and system is lost.
\begin{itemize}
\item \textbf{System uses odometric navigation}
\end{itemize}
\item Human: Due to the odometric sensors safely fixed in the car, crashes will not occur, making it much harder for the car to hit a person or for any part of the car to jump and cause harm to the user or anyone around.
\end{itemize}
\subsection{Image acquisition}
\label{sec:orgb6a5f66}
\subsubsection{Frame rate}
\label{sec:org5adf4ee}
Frame rate refers to the frequency at which independent still images appear on the screen. The higher the frame rate, a better image quality is obtained but the processing overhead increases as well, so a compromise must be achieved between the quality of the image and the processing overhead required.
\subsubsection{Range}
\label{sec:orgecb044c}
How far can the camera capture images without loosing resolution and record them.
\subsubsection{Resolution}
\label{sec:orgba87554}
The amount of detail that the camera can capture. It is measured in pixels. The quality of the aquired image is proportional to the number os pixels but a greater resolution requires a greater data transfer and processing overhead, thus, a compromise must be achieved.
\subsubsection{Color scale (Black and white or color)}
\label{sec:org468ee04}
É preciso estar aqui isto ?
\subsubsection{Always present or enabled on user command}
\label{sec:orgd585352}
É preciso estar aqui isto ?
\subsection{Usability}
\label{sec:org61632e0}
\begin{itemize}
\item User-friendly interface
\item User interface responsiveness
\end{itemize}
\subsection{Load}
\label{sec:orgca6a690}
The remote control car can be used 
\subsection{Overall System latency/responsivess}
\label{sec:org7fd1829}
The overall system latency is the sum of all systems' latencies, which must be
under a maximum tolerated value for the user.
\subsection{Communication}
\label{sec:org4241610}
\subsubsection{Reliability}
\label{sec:orgdcb920d}
Packet must be delivered (reliable, e.g. TCP) or not (e.g. UDP)
\subsubsection{Range}
\label{sec:org447a205}
The communication protocols have a limited range of operation, and, as such, regarding the environment on which the car is used the range can be changed.
The range refers to the maximum distance allowed between user and system for communication purposes.
\subsubsection{Transmission rate / throughput}
\label{sec:org10e75a5}
\subsubsection{Redundancy}
\label{sec:orgc5933fc}
The communication protocols are not flawless and the car relies on them to be controlled. If the communication is lost, the car cannot be controlled. A possible solution for this issue is using more communication protocols (e.g Wi-fi and bluetooth), so when one protocol fails, the car can still be controlled by the other.
\subsection{Sensibility}
\label{sec:org622e63a}
The movement of the car will be determined by the tilt movement of the smartphone. Sensibility refers to the responsiveness of the car on the minimum smartphone tilt movement.
\subsubsection{Msg Smartphone->Raspberry}
\label{sec:org6b5cb97}
x10 y20 v10
t5 v5

\noindent\rule{\textwidth}{0.5pt}
\subsection{Closed loop error (Control team)}
\label{sec:org436f732}
The closed loop control assures that upon the loss of communication or a command from the user that can cause harm to the car or anyone, the car will not crash. To do so, the velocity, direction and distance to objects must be controlled. The control consists on the comparation of the desired position with the current position, thus generation the error.
\subsubsection{PI}
\label{sec:org9859444}
\subsubsection{PID}
\label{sec:org352c4d4}
\subsubsection{PD}
\label{sec:org0d324c4}

\subsection{Summary}
\label{sec:org1f95256}
Table \ref{tab:specs-init} lista the foreseen product specifications.

% Please add the following required packages to your document preamble:
\begin{table}[!hbt]
\centering
\caption{Specifications}
\label{tab:specs-init}
\resizebox{\textwidth}{!}{%
\begin{tabular}{lll}
\hline
 & Values & Explanation \\ \hline
Max Velocity & 0.2 m/s & Maximum velocity of the conveyor belt in steady state \\ \hline
Dimensions & 60x30x30 cm & Dimensions of the conveyor belt in cm {[}l w h{]} \\ \hline
Time Min & 3 s & \begin{tabular}[c]{@{}l@{}}Time taken to transport a load the full extent \\ of the conveyor belt at maximum velocity\end{tabular} \\ \hline
Max Load & 1 Kg & \begin{tabular}[c]{@{}l@{}}Load the belt can hold without causing any \\ harm to the product\end{tabular} \\ \hline
Max slope & 15$^\circ$ & \begin{tabular}[c]{@{}l@{}}Maximum slope in which the conveyor belt can \\ operate at nominal conditions\end{tabular} \\ \hline
Slope levels & {[}0,5,10,15{]}$^\circ$ & Different levels of slope manually handled \\ \hline
Settling time & 0.2 $\cdot$ T & \begin{tabular}[c]{@{}l@{}}This means that it takes up to 20\% of the full \\ travel time to reach steady velocity\end{tabular} \\ \hline
Overshoot & 110\% Vss & \begin{tabular}[c]{@{}l@{}}Maximum velocity the conveyor belt reaches \\ before settling time\end{tabular} \\ \hline
Margin of error & 95\%-105\% of Vss & Admissible error in steady velocity \\ \hline
Power supply & 12V batteries, 6W & The main power supply will be 12V batteries \\ \hline
\end{tabular}%
}
\end{table}

%%% Local Variables:
%%% mode: latex
%%% TeX-master: "../Phase1"
%%% End:

%

%%% Local Variables:
%%% mode: latex
%%% TeX-master: "../../../dissertation"
%%% End:
