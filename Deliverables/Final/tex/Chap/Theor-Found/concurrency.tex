%      % Concurrency
\section{Concurrency}%
\label{sec:concurrency}
Concurrency is used to refer to things that appear to happen at the same time,
but which may occur serially~\cite{buttlar1996pthreads}, like the case of a multithreaded execution in
a single processor system.
Two concurrent tasks may start, execute and finish in overlapping instants of
time, without the two being executed at the same time.
As defined in POSIX, a concurrent execution requires that a function that
suspends the calling thread shall not suspend other threads, indefinitely.

This concept is different from parallelism. Parallelism refers to the
simultaneous execution of tasks, like the one of a multithreaded program in a
multiprocessor system.
Two parallel tasks are executed at the same time and, as such, they require
the execution in exclusivity in independent processors.

Every concurrent system provides three important facilities~\cite{buttlar1996pthreads}:
\begin{itemize}
\item \textbf{Execution Context}: refers to the concurrent entity state. It
  allows the context switch and it must maintain the entities states,
  independently.
\item \textbf{Scheduling}: in a concurrent system, the scheduling decides what
  context should execute at any given time.
\item \textbf{Synchronisation}: this allows the management of shared resources
  between the concurrent execution contexts.
\end{itemize}
%%% Local Variables:
%%% mode: latex
%%% TeX-master: "../../../dissertation"
%%% End:
