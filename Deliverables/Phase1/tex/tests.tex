\section{Tests}%
\label{sec:org3e2776f}
Tests are generally regarded as those performed over any physical
component or prototype. Here, however, a broader sense is used, to reflect the
tests conducted into the system and the several prototypes, under the abnormal
present circumstances.
Moreover, as indicated in the design, the current development
strategy encompasses the virtualization of all hardware components, enclosed
in a single virtual environment.

Thus, it does not make sense to perform
hardware related tests such as velocity measurements, autonomy, safety, etc. As
such, the focus is shifted towards software and control verification,
encompassing the following tests: functionality, image acquisition,
communication, and control algorithms correctness.

The tests are divided into verification and validation tests.
\subsection{Verification tests}%
\label{sec:orge9c79e2}
The verification tests are tests performed internally by the design team to
check the compliance of the foreseen specifications. These tests are done after
the prototype alpha is concluded.

\subsubsection{Functionality}%
\label{sec:functionality}
The remotely operated vehicle is composed of several modules distributed along
several different platforms, some of which distanced from each other. Thus, the
proposed sets of functionalities should be tested in the integrated system, by
tracking and analysing the user commands issued along the way until it finally
reaches the vehicle (in the virtual environment), assessing if it is correctly
processed. For example, if the user issues the vehicle to move to a given place
(via smartphone interaction), the message sent to the vehicle must be signaled
in each endpoint hit, and the vehicle should move to that place, symbolically
detected by the modification of its virtual coordinates.

%\subsubsection{Maximum Load}%
%\label{sec:load}
%As mentioned in Section~\ref{sec:load-specs}, the maximum load can be defined
%for the minimum speed (safe motor operation) and for maximum power consuptiom
%increase. Thus, two alternative definitions, and consequently, tests arise for the maximum load determination:
%\begin{enumerate}
%\item maximum load (at minimum speed): maximum load the vehicle can carry
%  safely at the minimum speed defined.
%\item maximum load (at 50\% over the mean power consumption): maximum load which
%  causes a 50\% increase in the mean power consumption, i.e., while operating at
%  mean speed.
%\end{enumerate}
%
%To test the former, load should be increased slowly, measuring the vehicle
%mean speed, until the minimum speed defined is achieved. To test the
%latter, load should be increased slowly, measuring the power consumption, until
%a 50\% increase over the mean power power consumption is detected, while
%operating at the mean speed.
%
%The maximum load can then be defined as the minimum one the vehicle can carry
%while observing both criteria.
%
%\subsubsection{Autonomy}%
%\label{sec:org532616f}
%The autonomy is related to product's power consumption and the capacity of the
%battery chosen. Under the present abnormal circumstances is not reasonable to
%expect the product's power consumption to match the real one, thus, for all
%purposes, this will be considered as the one drawn by the car module itself,
%namely, the installed motors and sensors.
%
%Then, the autonomy can be measured as the time interval between battery fully
%charged and safely discharged (the car stops), by fixating the car to a
%supporting structure with free moving wheels, and imposing the aforementioned
%conditions in Section~\ref{sec:autonomy-specs}.
%
%\subsubsection{Vehicle's Speed}%
%\label{sec:org20789b4}
%The vehicle must be operated within a safe range of speed, as mentioned in
%Section~\ref{speed-tests}.
%The optimum speed maximizes the energy efficiency, increasing battery's life and
%autonomy. The optimum speed must be previously defined through extensive
%vehicle's model simulation. Then, it can be tested experimentally: the vehicle
%must complete a trial track with a sufficiently long
%distance to assure speed reaching and stabilization. The experimental speed is
%then compared to the theorethical speed to measure the error.
%
%and compared to the ones provided in the foreseen specifications.
%The optimum speed maximizes the energy efficiency,
%accounting for the power losses associated with vehicle's dynamics (rolling
%friction, drag, etc.), as well as the increased temperature operation in the
%electronics components
%within a sufficiently long distance to assure speed reach and stabilization,
%and compared to the ones provided in the foreseen specifications.

%\subsubsection{Safety}%
%\label{sec:orgf4c025f}
%As mentioned in Section~\ref{sec:safety-tests}, safety can be analysed in two
%ways, considering the preservation of people and goods.
%
%To test human safety, it is important to identify the interactions between the
%user and the product, and which are the most prevalent and dangerous. Even so,
%the exhaustive test is outside the scope of the present work; a small set of
%features will be tested accordingly to the devised user manual, containing the
%safety measures. For example, battery installation and conditions should be
%tested, eventually leading to the posterior incorporation of safety measures in
%the product.
%
%To test goods safety, it is reasonable to assume the operating conditions of the
%vehicle. Under these it is important to consider the most critical ones that
%concern the moments when the vehicle is left to be controlled locally, instead
%of user controlled operation. The critical conditions for local operations are
%divided into two sets:
%\begin{itemize}
%\item processing of user commands and vehicle operation: user commands can
%  conflict with safety measures and, thus, should be overriden locally.
%\item communication loss: the vehicle is left to odometric navigation,
%  preserving the safety of people and goods.
%\end{itemize}
%To test these two scenarios, they should be replicated, observing the system
%response and tolerance.

\subsubsection{Image acquisition}%
\label{sec:orgb1f5c2a}
The vehicle is equipped with a camera to assist the user in its navigation,
thus, requiring the following variables to be tested: frame rate, range, and
resolution. In the current scenario, the virtual environment should provide
access to a built-in or external camera, with the former being fairly common
nowadays in every personal computer, thus, enabling easy testing.

\paragraph{Frame rate}%
\label{sec:frame-rate-test}
To test frame rate, the user screen must be updated with the number of frames
received from the camera per second and checked against the defined boundaries.

\paragraph{Range}%
\label{sec:range-test}
To test camera's range, an object must be captured at increasing distances, until the image resolution is lost.

\paragraph{Resolution}%
\label{sec:resolution-test}
The minimum resolution should be tested as providing the least amount of
information required for the user, while minimizing data transfer and processing overhead.

\subsubsection{Communication}%
\label{sec:comm-tests}
The communication tests are performed in compliance to the specifications
provided in Section~\ref{sec:org4241610}, namely reliability and redundancy.

\paragraph{Reliability}%
\label{sec:comm-reliability-test}
To test communication reliability, the most critical communication link is
chosen, namely the wireless communication between user's platform (smartphone)
and vehicle's platform (virtual environment). Then, the communication link and
protocol selected are tested by monitoring the ratio of dropped packets versus
total packets, using an appropriate tool on both directions for transmission and reception.

\paragraph{Redundancy}%
\label{sec:redundancy-test}
To test communication redundancy, one communication channel should be turned
off, verifying if the communication between nodes is still possible through
another communication channel. For example, in the communication between host
(user's platform) and remote (virtual environment) systems, the priority
communication is performed between host and high-level subsystem. However, if
this is turned off, the host must also be able to communicate with the remote
system via low-level subsystem.

\subsubsection{Correctness of the control algorithms}%
\label{sec:correct-control-algorithms-test}
As previously mentioned, the speed and position must be continuosly monitored
to ensure proper vehicle operation. Under the current circumstances, it is
difficult to properly and realistically stimulate the control loops, as both
sensor and actuator data are not available. The solution could reside in the
physical modeling of both actuator and sensor (the plant), as well of
controller's logic modeling, and perform formal verification
using model-checking and finite automata. Later on, those models could be tested
using software-in-loop (and hardware-in-loop, if hardware was available). 

However, physical modeling is not trivial, and it may still produce unrealistic
data. Thus, the solution found resides in external stimulation of the control
loops through input files containing the relevant data. Then, the behaviour of
the system can be analysed and verified for some cornercase situations,
assessing the control algorithms correctness.
%
%An overshoot occurs when the output in a control system exceeds its final,
%steady state value generally caused by a sudden change in the system, in this
%case specifically, the placement of a load upon the conveyor will cause an
%overshoot in the latter’s speed which must be controlled lest it cause
%problems.
%
%An overshoot will occur during the settling time, as such, using the same
%considerations taken in its measurement, it can be measured by observing the
%induced voltage at the generator (an overshoot in the conveyor’s speed will
%correspond to a peak in the generator’s voltage).
%
%Using an oscilloscope to display the induced voltage at the generator and making
%use of the “single mode” present in these measuring instruments one can observe
%the change that will occur in the generator’s induced voltage, the peak voltage
%that will be seen when the load is placed upon the running conveyor is the
%electrical representation of the overshoot of speed, then either by
%converting it to its physical representation or comparing it to the reference
%voltage one can arrive at a conclusion. It was agreed that the overshoot
%speed should be \(V_{ss} \pm 10\%\), where \(V_{ss}\) is the stedy state
%speed.

\subsection{Validation tests}%
\label{sec:orgff1a37d}
The validation tests should be performed by the client using the product’s
manual, so it is expected that a user without prior experience with the product
should be able to use it correctly and safely. On the present abnormal
circumstances, with limited access to the physical modules, specially for an
external agent, the validation is severely limited.
Thus, it should be limited to user interface validation.

For this purpose, an external agent will be provided with the software
application and the respective installation and usage manuals, and the feedback
will be collected and processed to further improve the product.
%%% Local Variables:
%%% mode: latex
%%% TeX-master: "../Phase1"
%%% End:
