\section{Tests}%
\label{sec:org3e2776f}
Tests are generally considered as those performed over any physical
component or prototype. Here, however, it is used in a broader sense, to reflect the tests
conducted into the system and the several prototypes, under the abnormal present
circumstances. The tests are divided into verification and validation tests.
\subsection{Verifications tests}%
\label{sec:orge9c79e2}
The verifications tests are tests performed internally by the design team to
check the compliance of the foreseen specifications. These tests are done after
the prototype alpha is concluded.

\subsubsection{Functionality}%
\label{sec:functionality}
The remotely operated vehicle is composed of several modules distributed along
several different platforms, some of which distanced from each other. In the
present abnormal circumstances, this is even more true. Thus, the proposed sets
of functionalities should be tested in the integrated system, by tracking and
analysing the user commands issued along the way until it finally reaches the
vehicle, assessing if it is correctly processed. For example, if the user issues
the vehicle to move to a given place, the message sent to the vehicle must be
signaled in each endpoint hit, and the vehicle should move to that place.

\subsubsection{Maximum Load}%
\label{sec:load}
The remotely controlled vehicle can be used in applications involving load
carrying (besides its own, obviously), e.g., packet delivery. For this purpose
it is important to determine the maximum load the vehicle can carry safely at
the minimum velocity defined. As the load increases, also increases the power
consumption, diminishing the autonomy. Thus, two alternative definitions, and
consequently, tests arise for the maximum load determination:
\begin{enumerate}
\item maximum load (at minimum velocity): maximum load the vehicle can carry
  safely at the minimum velocity defined.
\item maximum load (at 50\% over the mean power consumption): maximum load which
  causes a 50\% increase in the mean power consumption, i.e., while operating at
  mean velocity.
\end{enumerate}

To test the former, load should be increased slowly, measuring the vehicle
mean velocity, until the minimum velocity defined is achieved. To test the
latter, load should be increased slowly, measuring the power consumption, until
a 50\% increase over the mean power power consumption is detected, while
operating at the mean velocity.

\subsubsection{Autonomy}%
\label{sec:org532616f}
The vehicle is operated off-the-grid, thus, a portable power source must be
included. The autonomy --- time interval between battery fully charged and
safely discharged --- should be observed for the following scenarios:
\begin{enumerate}
\item No load and vehicle operating at maximum speed
\item No load and vehicle operating at mean speed
\item Maximum load and vehicle operating at maximum speed
\item Maximum load and vehicle operating at mean speed
\end{enumerate}
The autonomy is related to product's power consumption and the capacity of the
battery chosen. Under the present abnormal circumstances is not reasonable to
expect the product's power consumption to match the real one, thus, for all
purposes, this will be considered as the one drawn by the car module itself,
namely, the installed motors and sensors.

Then, the autonomy can be measured as
the time interval between battery fully charged and safely discharged (the car
stops), by fixating the car to a supporting structure with free moving wheels,
and imposing the aforementioned conditions.

\subsubsection{Velocity}%
\label{sec:org20789b4}
The vehicle must be operated within a safe range of velocity, while also not
increasing excessively the power consumption. Thus, these velocity boundaries
should be tested in the absence of an external load and in the presence of the
maximum load. It is important to define these boundaries as follows:
\begin{itemize}
\item minimum velocity: minimum velocity defined for the vehicle, which must be
  attained even in the condition of maximum load. This value must be selected to
  assure safe motor operation.
\item maximum velocity (no load): maximum velocity for the vehicle in the
  absence of an external load. This is the absolute maximum velocity for the
  vehicle.
\item maximum velocity (maximum load): maximum velocity for the vehicle in the
  presence of the maximum load. This value must be selected to prevent excessive
  power consumption and motor overheating.
\end{itemize}
The aforementioned velocities should be tested in the designed conditions,
within a sufficiently long distance to assure velocity reach and stabilization,
and compared to the ones provided in the foreseen specifications.

\subsubsection{Safety}%
\label{sec:orgf4c025f}
Safety is paramount in product design, especially considering the vehicle is to
be remotely operated. Safety can be analysed in two ways, considering the
preservation of people and goods. For the former, it is important to assure safe
user operation as well as safe human interaction --- the vehicle may encounter
several people along its path, but it must not inflict any damage. For the
latter, the vehicle under operating conditions must not inflict any damage to
goods.

To test human safety, it is important to identify the interactions between the
user and the product, and which are the most prevalent and dangerous. Even so,
the exhaustive test is outside the scope of the present work; a small set of
features will be tested accordingly to the devised user manual, containing the
safety measures. For example, battery installation and conditions should be
tested, eventually leading to the posterior incorporation of safety measures in
the product.

To test goods safety, it is reasonable to assume the operating conditions of the
vehicle. Under these it is important to consider the most critical ones that
concern the moments when the vehicle is left to be controlled locally, instead
of user controlled operation. The critical conditions for local operations are
divided into two sets:
\begin{itemize}
\item processing of user commands and vehicle operation: user commands can
  conflict with safety measures and, thus, should be overriden locally.
\item communication loss: the vehicle is left to odometric navigation,
  preserving the safety of people and goods.
\end{itemize}
To test these two scenarios, they should be replicated, observing the system
response and tolerance.%DONE

\subsubsection{Image acquisition}%
\label{sec:orgb1f5c2a}
The vehicle is equipped with a camera to assist the user in its navigation,
thus, requiring it to be feed to the user's platform within proper conditions.
The following variables are to be tested: frame rate, range, and resolution.

\paragraph{Frame rate}%
\label{sec:frame-rate-test}
The frame rate is the rate at which the user platform screen is updated with new
image information. It should be maintained within acceptable boundaries to serve
the purpose of assisting the user in the vehicle's navigation. To test it, the
number of frame displayed per second in the user screen must be updated and
checked against the defined boundaries.

\paragraph{Range}%
\label{sec:range-test}
The range is the maximum distance the camera can clearly effectively capture an
image without losing resolution. To test this, an object must be captured at
increasing distances, until the image resolution is lost.

\paragraph{Resolution}%
\label{sec:resolution-test}
The image resolution quantifies how close lines can be to each other and still
be visibly resolved, giving an information on the its detail. The minimum
resolution should be tested as providing the least amount of information
required for the user, while minimizing data transfer and processing overhead.

\subsubsection{Communication reliability}%
\label{sec:comm-reli}
A communication is reliable if it guarantees measures to deliver the data
conveyed in the communication link. As reliability imposes these measures, it
also adds overhead to the communication protocol, which must be considered
depending on the case. For example, for the devised product, an user command
must be acknowledged to be processed, otherwise, the user must be informed; on
the other hand, loosing frames from the video feed is not so critical --- user
can still observe conveniently the field of vision if the frame rate is within
acceptable boundaries. 

Thus, given the critical nature of user commands issued, the focus will be on
this communication link. To test the reliability dummy packets should be sent
from the user platform to the vehicle and be acknowledged and parsed correctly.

\subsubsection{Closed loop error (Control loops)}%
\label{sec:closed-loop-error}
The velocity, direction and distance to obstacles must be continuosly monitored
to ensure proper vehicle operation. The closed loop error must then be checked
mainly in three situations as a response to an user command:
\begin{itemize}
\item velocity: the user issued an command with a given mean velocity, which
  should be compared with the steady-state mean velocity of the vehicle. This
  can be tested by comparing the user defined velocity and the vehicle's;
\item direction: the user issued an command with a given direction, which should
  be compared to the vehicle direction. This can be tested by measuring the
  angle between final and initial points and comparing it with the user defined
  direction.
\item distance to obstacles: the user issued an command with a given direction
  and velocity which can cause it to crash. The local control must take over
  control, preventing this to happen, and the final distance to the obstacles
  must be assessed and compared to the defined one.
\end{itemize}

%An overshoot occurs when the output in a control system exceeds its final,
%steady state value generally caused by a sudden change in the system, in this
%case specifically, the placement of a load upon the conveyor will cause an
%overshoot in the latter’s velocity which must be controlled lest it cause
%problems.
%
%An overshoot will occur during the settling time, as such, using the same
%considerations taken in its measurement, it can be measured by observing the
%induced voltage at the generator (an overshoot in the conveyor’s velocity will
%correspond to a peak in the generator’s voltage).
%
%Using an oscilloscope to display the induced voltage at the generator and making
%use of the “single mode” present in these measuring instruments one can observe
%the change that will occur in the generator’s induced voltage, the peak voltage
%that will be seen when the load is placed upon the running conveyor is the
%electrical representation of the overshoot of velocity, then either by
%converting it to its physical representation or comparing it to the reference
%voltage one can arrive at a conclusion. It was agreed that the overshoot
%velocity should be \(V_{ss} \pm 10\%\), where \(V_{ss}\) is the stedy state
%velocity.

\subsection{Validation tests}%
\label{sec:orgff1a37d}
The validation tests should be performed by the client using the product’s
manual, so it is expected that a user without prior experience with the product
should be able to use it correctly and safely. On the present abnormal
circumstances, with limited access to the physical modules, specially for an
external agent, the validation is severely limited.
Thus, it should be limited to user interface validation.

For this purpose, an external agent will be provided with the software
application and the respective installation and usage manuals, and the feedback
will be collected and processed to further improve the product.
%%% Local Variables:
%%% mode: latex
%%% TeX-master: "../Phase1"
%%% End:
